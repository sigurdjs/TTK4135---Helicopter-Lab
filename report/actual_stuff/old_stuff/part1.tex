\section{Problem Description}\label{sec:prob_descr}
You should have a section that describes the lab setup, including a model of the helicopter. If you want, you can copy the source code for the model equations:
\begin{gather}
	\ddot{e} + K_{3} K_{ed} \dot{e} + K_{3} K_{ep} e = K_{3} K_{ep} e_{c} \label{eq:model_elev} \\
	\ddot{p} + K_{1} K_{pd} \dot{p} + K_{1} K_{pp} p = K_{1} K_{pp} p_{c} \label{eq:model_pitch} \\
	\dot{\lambda} = r \label{eq:model_lambda} \\
	\dot{r} = -K_{2} p \label{eq:model_r} 
\end{gather}
Since these equations belong together, it's a good idea to number them like this:

You can then both reference individual equations (``the elevation equation \Cref{eq:model_se_elev}'') or reference the entire model 



\subsection{Illustrations}
If you decide to include an illustration, that's great. You can in general copy figures and illustrations from the textbook, the assignement text, or other places. However: ALWAYS CITE THE SOURCE\@. \cite{Nocedal2006} You can also draw your own (cite the source if it is heavily based on someone else's.). \Cref{fig:layers_openloop} was created quickly with Ipe (\url{http://ipe.otfried.org/}). Inkscape is a good alternative for more advanced illustrations. Some people prefer the Latex package TikZ (\url{http://texample.net/tikz/examples/}), but this takes a little effort to learn.


\begin{figure}
    \centering
    % This file was created by matlab2tikz.
%
%The latest updates can be retrieved from
%  http://www.mathworks.com/matlabcentral/fileexchange/22022-matlab2tikz-matlab2tikz
%where you can also make suggestions and rate matlab2tikz.
%
\begin{tikzpicture}

\begin{axis}[%
width=4.602in,
height=3.82in,
at={(0.772in,0.516in)},
scale only axis,
xmin=0,
xmax=30,
xlabel={Time[s]},
ymin=-0.6,
ymax=0.6,
ylabel={Elevation[rad]},
axis background/.style={fill=white},
legend style={legend cell align=left, align=left, draw=black}
]
\addplot [color=blue]
  table[row sep=crcr]{%
0	0\\
5	0\\
5.25	0.0111790084870798\\
5.5	0.0314285228123445\\
5.75	0.0589571543992378\\
6	0.0922242778688513\\
6.25	0.129875847906554\\
6.5	0.170685381629585\\
6.75	0.213497796741997\\
7.25	0.300541641256594\\
7.5	0.34232898323306\\
7.75	0.38111598941132\\
8	0.415265459007205\\
8.25	0.442852238150678\\
8.5	0.46158524270497\\
8.75	0.468715878925309\\
9	0.466851842459036\\
9.25	0.458129971010763\\
9.5	0.444293585959862\\
9.75	0.426757677384973\\
10	0.406662971231832\\
10.25	0.384923307218362\\
10.75	0.339254788123753\\
11.25	0.293847504952478\\
11.5	0.272037456697252\\
11.75	0.251088367331022\\
12	0.231125940976632\\
12.25	0.212230920522753\\
12.5	0.194448554601916\\
12.75	0.177795884159032\\
13	0.16226822316548\\
13.25	0.147844447077212\\
13.5	0.134491694237489\\
13.75	0.122167453715186\\
14	0.110823445674527\\
14.25	0.100407024401854\\
14.5	0.0908636264836069\\
14.75	0.0821376572517032\\
15	0.0741736887539624\\
15.25	0.0669178221019777\\
15.5	0.0603172853547704\\
15.75	0.0543215239931385\\
16	0.0488827401906597\\
16.25	0.0439552884880605\\
16.5	0.039496113114545\\
17	0.0318252011785489\\
17.5	0.025581216082383\\
18	0.0205163610018566\\
18.5	0.0164208238337658\\
19	0.0131176017427954\\
19.75	0.00933514693502246\\
20.5	0.00662072174521811\\
21.25	0.00468070626050476\\
22.25	0.00293462057759797\\
23.5	0.00162741815917045\\
24.75	0.000897979831691487\\
25	0\\
29.75	0\\
};
\addlegendentry{$\text{e}_{\text{calculated}}$}

\addplot [color=black!50!green]
  table[row sep=crcr]{%
0	-0.523598775598298\\
0.166	-0.523598775598298\\
0.167999999999999	-0.522064794810412\\
0.218	-0.522064794810412\\
0.219999999999999	-0.520530814022528\\
0.25	-0.520530814022528\\
0.251999999999999	-0.518996833234642\\
0.27	-0.518996833234642\\
0.271999999999998	-0.517462852446755\\
0.289999999999999	-0.517462852446755\\
0.292000000000002	-0.515928871658872\\
0.303999999999998	-0.515928871658872\\
0.306000000000001	-0.514394890870985\\
0.315999999999999	-0.514394890870985\\
0.318000000000001	-0.512860910083099\\
0.327999999999999	-0.512860910083099\\
0.329999999999998	-0.511326929295215\\
0.338000000000001	-0.511326929295215\\
0.34	-0.509792948507329\\
0.347999999999999	-0.509792948507329\\
0.350000000000001	-0.508258967719442\\
0.358000000000001	-0.508258967719442\\
0.359999999999999	-0.506724986931555\\
0.366	-0.506724986931555\\
0.367999999999999	-0.505191006143672\\
0.373999999999999	-0.505191006143672\\
0.376000000000001	-0.503657025355785\\
0.382000000000001	-0.503657025355785\\
0.384	-0.502123044567899\\
0.390000000000001	-0.502123044567899\\
0.391999999999999	-0.500589063780016\\
0.396000000000001	-0.500589063780016\\
0.398	-0.499055082992129\\
0.402000000000001	-0.499055082992129\\
0.404	-0.497521102204242\\
0.41	-0.497521102204242\\
0.411999999999999	-0.495987121416356\\
0.416	-0.495987121416356\\
0.417999999999999	-0.494453140628472\\
0.422000000000001	-0.494453140628472\\
0.423999999999999	-0.492919159840586\\
0.425999999999998	-0.492919159840586\\
0.428000000000001	-0.491385179052699\\
0.431999999999999	-0.491385179052699\\
0.434000000000001	-0.489851198264816\\
0.437999999999999	-0.489851198264816\\
0.440000000000001	-0.488317217476929\\
0.442	-0.488317217476929\\
0.443999999999999	-0.486783236689043\\
0.448	-0.486783236689043\\
0.449999999999999	-0.485249255901159\\
0.452000000000002	-0.485249255901159\\
0.454000000000001	-0.483715275113273\\
0.457999999999998	-0.483715275113273\\
0.460000000000001	-0.482181294325386\\
0.462	-0.482181294325386\\
0.463999999999999	-0.480647313537499\\
0.466000000000001	-0.480647313537499\\
0.468	-0.479113332749616\\
0.469999999999999	-0.479113332749616\\
0.472000000000001	-0.47757935196173\\
0.475999999999999	-0.47757935196173\\
0.478000000000002	-0.476045371173843\\
0.48	-0.476045371173843\\
0.481999999999999	-0.47451139038596\\
0.484000000000002	-0.47451139038596\\
0.486000000000001	-0.472977409598073\\
0.488	-0.472977409598073\\
0.489999999999998	-0.471443428810186\\
0.492000000000001	-0.471443428810186\\
0.494	-0.469909448022303\\
0.495999999999999	-0.469909448022303\\
0.498000000000001	-0.468375467234416\\
0.5	-0.468375467234416\\
0.501999999999999	-0.46684148644653\\
0.504000000000001	-0.46684148644653\\
0.506	-0.465307505658643\\
0.507999999999999	-0.465307505658643\\
0.512	-0.462239544082873\\
0.513999999999999	-0.462239544082873\\
0.516000000000002	-0.460705563294987\\
0.518000000000001	-0.460705563294987\\
0.52	-0.459171582507103\\
0.521999999999998	-0.459171582507103\\
0.526	-0.45610362093133\\
0.527999999999999	-0.45610362093133\\
0.530000000000001	-0.454569640143443\\
0.532	-0.454569640143443\\
0.533999999999999	-0.45303565935556\\
0.536000000000001	-0.45303565935556\\
0.539999999999999	-0.449967697779787\\
0.542000000000002	-0.449967697779787\\
0.544	-0.448433716991904\\
0.545999999999999	-0.448433716991904\\
0.550000000000001	-0.44536575541613\\
0.552	-0.44536575541613\\
0.556000000000001	-0.44229779384036\\
0.558	-0.44229779384036\\
0.559999999999999	-0.440763813052474\\
0.562000000000001	-0.440763813052474\\
0.568000000000001	-0.436161870688817\\
0.57	-0.436161870688817\\
0.571999999999999	-0.434627889900931\\
0.574000000000002	-0.434627889900931\\
0.579999999999998	-0.430025947537274\\
0.582000000000001	-0.430025947537274\\
0.588000000000001	-0.425424005173618\\
0.59	-0.425424005173618\\
0.594000000000001	-0.422356043597848\\
0.596	-0.422356043597848\\
0.602	-0.417754101234191\\
0.603999999999999	-0.417754101234191\\
0.611999999999998	-0.411618178082648\\
0.614000000000001	-0.411618178082648\\
0.622	-0.405482254931105\\
0.623999999999999	-0.405482254931105\\
0.629999999999999	-0.400880312567448\\
0.632000000000001	-0.400880312567448\\
0.641999999999999	-0.393210408628018\\
0.643999999999998	-0.393210408628018\\
0.652000000000001	-0.387074485476475\\
0.654	-0.387074485476475\\
0.661999999999999	-0.380938562324936\\
0.664000000000001	-0.380938562324936\\
0.673999999999999	-0.373268658385506\\
0.675999999999998	-0.373268658385506\\
0.692	-0.360996812082419\\
0.693999999999999	-0.360996812082419\\
0.707999999999998	-0.350258946567223\\
0.710000000000001	-0.350258946567223\\
0.728000000000002	-0.33645311947625\\
0.73	-0.33645311947625\\
0.75	-0.321113311597394\\
0.751999999999999	-0.321113311597394\\
0.77	-0.307307484506424\\
0.771999999999998	-0.307307484506424\\
0.788	-0.295035638203338\\
0.789999999999999	-0.295035638203338\\
0.806000000000001	-0.282763791900255\\
0.808	-0.282763791900255\\
0.821999999999999	-0.272025926385055\\
0.824000000000002	-0.272025926385055\\
0.835999999999999	-0.262822041657738\\
0.838000000000001	-0.262822041657738\\
0.850000000000001	-0.253618156930425\\
0.852	-0.253618156930425\\
0.861999999999998	-0.245948252990999\\
0.864000000000001	-0.245948252990999\\
0.876000000000001	-0.236744368263683\\
0.878	-0.236744368263683\\
0.948	-0.192258925415\\
0.949999999999999	-0.192258925415\\
0.957999999999998	-0.186123002263457\\
0.960000000000001	-0.186123002263457\\
0.963999999999999	-0.183055040687687\\
0.966000000000001	-0.183055040687687\\
0.972000000000001	-0.178453098324031\\
0.974	-0.178453098324031\\
0.98	-0.173851155960374\\
0.981999999999999	-0.173851155960374\\
0.988	-0.169249213596714\\
0.989999999999998	-0.169249213596714\\
0.994	-0.166181252020944\\
0.995999999999999	-0.166181252020944\\
1.002	-0.161579309657288\\
1.004	-0.161579309657288\\
1.01	-0.156977367293631\\
1.012	-0.156977367293631\\
1.016	-0.153909405717858\\
1.018	-0.153909405717858\\
1.024	-0.149307463354202\\
1.026	-0.149307463354202\\
1.03	-0.146239501778432\\
1.032	-0.146239501778432\\
1.036	-0.143171540202658\\
1.038	-0.143171540202658\\
1.042	-0.140103578626888\\
1.044	-0.140103578626888\\
1.05	-0.135501636263232\\
1.052	-0.135501636263232\\
1.056	-0.132433674687462\\
1.058	-0.132433674687462\\
1.06	-0.130899693899575\\
1.062	-0.130899693899575\\
1.066	-0.127831732323802\\
1.068	-0.127831732323802\\
1.072	-0.124763770748032\\
1.074	-0.124763770748032\\
1.078	-0.121695809172262\\
1.08	-0.121695809172262\\
1.082	-0.120161828384376\\
1.084	-0.120161828384376\\
1.088	-0.117093866808602\\
1.09	-0.117093866808602\\
1.094	-0.114025905232833\\
1.096	-0.114025905232833\\
1.098	-0.112491924444946\\
1.1	-0.112491924444946\\
1.102	-0.110957943657063\\
1.104	-0.110957943657063\\
1.108	-0.107889982081289\\
1.11	-0.107889982081289\\
1.114	-0.104822020505519\\
1.116	-0.104822020505519\\
1.118	-0.103288039717633\\
1.12	-0.103288039717633\\
1.122	-0.101754058929746\\
1.124	-0.101754058929746\\
1.128	-0.0986860973539763\\
1.13	-0.0986860973539763\\
1.132	-0.0971521165660896\\
1.134	-0.0971521165660896\\
1.136	-0.0956181357782064\\
1.138	-0.0956181357782064\\
1.142	-0.0925501742024331\\
1.144	-0.0925501742024331\\
1.146	-0.0910161934145464\\
1.148	-0.0910161934145464\\
1.15	-0.0894822126266632\\
1.152	-0.0894822126266632\\
1.154	-0.0879482318387765\\
1.156	-0.0879482318387765\\
1.16	-0.0848802702630067\\
1.162	-0.0848802702630067\\
1.164	-0.08334628947512\\
1.166	-0.08334628947512\\
1.168	-0.0818123086872333\\
1.17	-0.0818123086872333\\
1.172	-0.0802783278993502\\
1.174	-0.0802783278993502\\
1.176	-0.0787443471114635\\
1.178	-0.0787443471114635\\
1.18	-0.0772103663235768\\
1.182	-0.0772103663235768\\
1.184	-0.0756763855356901\\
1.186	-0.0756763855356901\\
1.188	-0.074142404747807\\
1.19	-0.074142404747807\\
1.192	-0.0726084239599203\\
1.194	-0.0726084239599203\\
1.196	-0.0710744431720336\\
1.198	-0.0710744431720336\\
1.2	-0.0695404623841505\\
1.202	-0.0695404623841505\\
1.204	-0.0680064815962638\\
1.206	-0.0680064815962638\\
1.208	-0.0664725008083771\\
1.21	-0.0664725008083771\\
1.212	-0.0649385200204904\\
1.216	-0.0649385200204904\\
1.218	-0.0634045392326072\\
1.22	-0.0634045392326072\\
1.222	-0.0618705584447206\\
1.224	-0.0618705584447206\\
1.226	-0.0603365776568339\\
1.228	-0.0603365776568339\\
1.23	-0.0588025968689507\\
1.234	-0.0588025968689507\\
1.236	-0.057268616081064\\
1.238	-0.057268616081064\\
1.24	-0.0557346352931773\\
1.242	-0.0557346352931773\\
1.244	-0.0542006545052942\\
1.248	-0.0542006545052942\\
1.25	-0.0526666737174075\\
1.252	-0.0526666737174075\\
1.254	-0.0511326929295208\\
1.258	-0.0511326929295208\\
1.26	-0.0495987121416341\\
1.262	-0.0495987121416341\\
1.264	-0.048064731353751\\
1.268	-0.048064731353751\\
1.27	-0.0465307505658643\\
1.274	-0.0465307505658643\\
1.276	-0.0449967697779776\\
1.278	-0.0449967697779776\\
1.28	-0.0434627889900945\\
1.284	-0.0434627889900945\\
1.286	-0.0419288082022078\\
1.29	-0.0419288082022078\\
1.292	-0.0403948274143211\\
1.296	-0.0403948274143211\\
1.298	-0.038860846626438\\
1.302	-0.038860846626438\\
1.304	-0.0373268658385513\\
1.308	-0.0373268658385513\\
1.31	-0.0357928850506646\\
1.314	-0.0357928850506646\\
1.316	-0.0342589042627779\\
1.32	-0.0342589042627779\\
1.322	-0.0327249234748948\\
1.328	-0.0327249234748948\\
1.33	-0.0311909426870081\\
1.334	-0.0311909426870081\\
1.336	-0.0296569618991214\\
1.34	-0.0296569618991214\\
1.342	-0.0281229811112382\\
1.348	-0.0281229811112382\\
1.35	-0.0265890003233515\\
1.356	-0.0265890003233515\\
1.358	-0.0250550195354649\\
1.364	-0.0250550195354649\\
1.366	-0.0235210387475782\\
1.374	-0.0235210387475782\\
1.376	-0.021987057959695\\
1.38	-0.021987057959695\\
1.382	-0.0204530771718083\\
1.392	-0.0204530771718083\\
1.394	-0.0189190963839216\\
1.402	-0.0189190963839216\\
1.404	-0.0173851155960385\\
1.414	-0.0173851155960385\\
1.416	-0.0158511348081518\\
1.424	-0.0158511348081518\\
1.426	-0.0143171540202651\\
1.438	-0.0143171540202651\\
1.44	-0.012783173232382\\
1.45	-0.012783173232382\\
1.452	-0.0112491924444953\\
1.468	-0.0112491924444953\\
1.47	-0.0097152116566086\\
1.49	-0.0097152116566086\\
1.492	-0.00818123086872191\\
1.53	-0.00818123086872191\\
1.532	-0.00664725008083877\\
1.602	-0.00664725008083877\\
1.604	-0.00818123086872191\\
1.632	-0.00818123086872191\\
1.634	-0.0097152116566086\\
1.652	-0.0097152116566086\\
1.654	-0.0112491924444953\\
1.672	-0.0112491924444953\\
1.674	-0.012783173232382\\
1.69	-0.012783173232382\\
1.692	-0.0143171540202651\\
1.706	-0.0143171540202651\\
1.708	-0.0158511348081518\\
1.72	-0.0158511348081518\\
1.722	-0.0173851155960385\\
1.734	-0.0173851155960385\\
1.736	-0.0189190963839216\\
1.744	-0.0189190963839216\\
1.746	-0.0204530771718083\\
1.756	-0.0204530771718083\\
1.758	-0.021987057959695\\
1.766	-0.021987057959695\\
1.768	-0.0235210387475782\\
1.778	-0.0235210387475782\\
1.78	-0.0250550195354649\\
1.788	-0.0250550195354649\\
1.79	-0.0265890003233515\\
1.8	-0.0265890003233515\\
1.802	-0.0281229811112382\\
1.81	-0.0281229811112382\\
1.812	-0.0296569618991214\\
1.82	-0.0296569618991214\\
1.822	-0.0311909426870081\\
1.832	-0.0311909426870081\\
1.834	-0.0327249234748948\\
1.842	-0.0327249234748948\\
1.844	-0.0342589042627779\\
1.852	-0.0342589042627779\\
1.854	-0.0357928850506646\\
1.862	-0.0357928850506646\\
1.864	-0.0373268658385513\\
1.874	-0.0373268658385513\\
1.876	-0.038860846626438\\
1.884	-0.038860846626438\\
1.886	-0.0403948274143211\\
1.894	-0.0403948274143211\\
1.896	-0.0419288082022078\\
1.904	-0.0419288082022078\\
1.906	-0.0434627889900945\\
1.914	-0.0434627889900945\\
1.916	-0.0449967697779776\\
1.926	-0.0449967697779776\\
1.928	-0.0465307505658643\\
1.936	-0.0465307505658643\\
1.938	-0.048064731353751\\
1.95	-0.048064731353751\\
1.952	-0.0495987121416341\\
1.962	-0.0495987121416341\\
1.964	-0.0511326929295208\\
1.976	-0.0511326929295208\\
1.978	-0.0526666737174075\\
1.99	-0.0526666737174075\\
1.992	-0.0542006545052942\\
2.008	-0.0542006545052942\\
2.01	-0.0557346352931773\\
2.028	-0.0557346352931773\\
2.03	-0.057268616081064\\
2.048	-0.057268616081064\\
2.05	-0.0588025968689507\\
2.068	-0.0588025968689507\\
2.07	-0.0603365776568339\\
2.104	-0.0603365776568339\\
2.106	-0.0618705584447206\\
2.216	-0.0618705584447206\\
2.218	-0.0603365776568339\\
2.248	-0.0603365776568339\\
2.25	-0.0588025968689507\\
2.276	-0.0588025968689507\\
2.278	-0.057268616081064\\
2.3	-0.057268616081064\\
2.302	-0.0557346352931773\\
2.32	-0.0557346352931773\\
2.322	-0.0542006545052942\\
2.334	-0.0542006545052942\\
2.336	-0.0526666737174075\\
2.352	-0.0526666737174075\\
2.354	-0.0511326929295208\\
2.364	-0.0511326929295208\\
2.366	-0.0495987121416341\\
2.376	-0.0495987121416341\\
2.378	-0.048064731353751\\
2.388	-0.048064731353751\\
2.39	-0.0465307505658643\\
2.404	-0.0465307505658643\\
2.406	-0.0449967697779776\\
2.416	-0.0449967697779776\\
2.418	-0.0434627889900945\\
2.426	-0.0434627889900945\\
2.428	-0.0419288082022078\\
2.436	-0.0419288082022078\\
2.438	-0.0403948274143211\\
2.448	-0.0403948274143211\\
2.45	-0.038860846626438\\
2.46	-0.038860846626438\\
2.462	-0.0373268658385513\\
2.47	-0.0373268658385513\\
2.472	-0.0357928850506646\\
2.48	-0.0357928850506646\\
2.482	-0.0342589042627779\\
2.49	-0.0342589042627779\\
2.492	-0.0327249234748948\\
2.502	-0.0327249234748948\\
2.504	-0.0311909426870081\\
2.512	-0.0311909426870081\\
2.514	-0.0296569618991214\\
2.524	-0.0296569618991214\\
2.526	-0.0281229811112382\\
2.536	-0.0281229811112382\\
2.538	-0.0265890003233515\\
2.548	-0.0265890003233515\\
2.55	-0.0250550195354649\\
2.56	-0.0250550195354649\\
2.562	-0.0235210387475782\\
2.572	-0.0235210387475782\\
2.574	-0.021987057959695\\
2.584	-0.021987057959695\\
2.586	-0.0204530771718083\\
2.596	-0.0204530771718083\\
2.598	-0.0189190963839216\\
2.61	-0.0189190963839216\\
2.612	-0.0173851155960385\\
2.624	-0.0173851155960385\\
2.626	-0.0158511348081518\\
2.64	-0.0158511348081518\\
2.642	-0.0143171540202651\\
2.658	-0.0143171540202651\\
2.66	-0.012783173232382\\
2.676	-0.012783173232382\\
2.678	-0.0112491924444953\\
2.694	-0.0112491924444953\\
2.696	-0.0097152116566086\\
2.718	-0.0097152116566086\\
2.72	-0.00818123086872191\\
2.75	-0.00818123086872191\\
2.752	-0.00664725008083877\\
2.82	-0.00664725008083877\\
2.822	-0.00511326929295208\\
2.85	-0.00511326929295208\\
2.852	-0.00664725008083877\\
2.908	-0.00664725008083877\\
2.91	-0.00818123086872191\\
2.952	-0.00818123086872191\\
2.954	-0.0097152116566086\\
2.982	-0.0097152116566086\\
2.984	-0.0112491924444953\\
3.014	-0.0112491924444953\\
3.016	-0.012783173232382\\
3.056	-0.012783173232382\\
3.058	-0.0143171540202651\\
3.126	-0.0143171540202651\\
3.128	-0.0158511348081518\\
3.152	-0.0158511348081518\\
3.154	-0.0143171540202651\\
3.216	-0.0143171540202651\\
3.218	-0.012783173232382\\
3.246	-0.012783173232382\\
3.248	-0.0112491924444953\\
3.27	-0.0112491924444953\\
3.272	-0.0097152116566086\\
3.294	-0.0097152116566086\\
3.296	-0.00818123086872191\\
3.316	-0.00818123086872191\\
3.318	-0.00664725008083877\\
3.336	-0.00664725008083877\\
3.338	-0.00511326929295208\\
3.356	-0.00511326929295208\\
3.358	-0.00357928850506539\\
3.374	-0.00357928850506539\\
3.376	-0.00204530771718225\\
3.392	-0.00204530771718225\\
3.394	-0.000511326929295564\\
3.414	-0.000511326929295564\\
3.416	0.00102265385859113\\
3.438	0.00102265385859113\\
3.44	0.00255663464647782\\
3.46	0.00255663464647782\\
3.462	0.00409061543436096\\
3.486	0.00409061543436096\\
3.488	0.00562459622224765\\
3.518	0.00562459622224765\\
3.52	0.00715857701013434\\
3.574	0.00715857701013434\\
3.576	0.00869255779801748\\
3.662	0.00869255779801748\\
3.664	0.00715857701013434\\
3.726	0.00715857701013434\\
3.728	0.00562459622224765\\
3.768	0.00562459622224765\\
3.77	0.00409061543436096\\
3.804	0.00409061543436096\\
3.806	0.00255663464647782\\
3.844	0.00255663464647782\\
3.846	0.00102265385859113\\
3.892	0.00102265385859113\\
3.894	-0.000511326929295564\\
4.066	-0.000511326929295564\\
4.068	0.00102265385859113\\
4.108	0.00102265385859113\\
4.11	0.00255663464647782\\
4.138	0.00255663464647782\\
4.14	0.00409061543436096\\
4.17	0.00409061543436096\\
4.172	0.00562459622224765\\
4.202	0.00562459622224765\\
4.204	0.00715857701013434\\
4.234	0.00715857701013434\\
4.236	0.00869255779801748\\
4.264	0.00869255779801748\\
4.266	0.0102265385859042\\
4.302	0.0102265385859042\\
4.304	0.0117605193737909\\
4.376	0.0117605193737909\\
4.378	0.013294500161674\\
4.452	0.013294500161674\\
4.454	0.0117605193737909\\
4.516	0.0117605193737909\\
4.518	0.0102265385859042\\
4.56	0.0102265385859042\\
4.562	0.00869255779801748\\
4.596	0.00869255779801748\\
4.598	0.00715857701013434\\
4.632	0.00715857701013434\\
4.634	0.00562459622224765\\
4.68	0.00562459622224765\\
4.682	0.00409061543436096\\
4.74	0.00409061543436096\\
4.742	0.00255663464647782\\
4.81	0.00255663464647782\\
4.812	0.00409061543436096\\
4.878	0.00409061543436096\\
4.88	0.00562459622224765\\
4.912	0.00562459622224765\\
4.914	0.00715857701013434\\
4.94	0.00715857701013434\\
4.942	0.00869255779801748\\
4.966	0.00869255779801748\\
4.968	0.0102265385859042\\
4.986	0.0102265385859042\\
4.988	0.0117605193737909\\
5.008	0.0117605193737909\\
5.01	0.013294500161674\\
5.026	0.013294500161674\\
5.028	0.0148284809495607\\
5.044	0.0148284809495607\\
5.046	0.0163624617374474\\
5.062	0.0163624617374474\\
5.064	0.0178964425253341\\
5.078	0.0178964425253341\\
5.08	0.0194304233132172\\
5.096	0.0194304233132172\\
5.098	0.0209644041011039\\
5.11	0.0209644041011039\\
5.112	0.0224983848889906\\
5.128	0.0224983848889906\\
5.13	0.0240323656768737\\
5.144	0.0240323656768737\\
5.146	0.0255663464647604\\
5.16	0.0255663464647604\\
5.162	0.0271003272526471\\
5.176	0.0271003272526471\\
5.178	0.0286343080405302\\
5.192	0.0286343080405302\\
5.194	0.0301682888284169\\
5.206	0.0301682888284169\\
5.208	0.0317022696163036\\
5.222	0.0317022696163036\\
5.224	0.0332362504041903\\
5.238	0.0332362504041903\\
5.24	0.0347702311920735\\
5.254	0.0347702311920735\\
5.256	0.0363042119799601\\
5.27	0.0363042119799601\\
5.272	0.0378381927678468\\
5.286	0.0378381927678468\\
5.288	0.03937217355573\\
5.302	0.03937217355573\\
5.304	0.0409061543436167\\
5.316	0.0409061543436167\\
5.318	0.0424401351315034\\
5.332	0.0424401351315034\\
5.334	0.04397411591939\\
5.346	0.04397411591939\\
5.348	0.0455080967072732\\
5.362	0.0455080967072732\\
5.364	0.0470420774951599\\
5.376	0.0470420774951599\\
5.378	0.0485760582830466\\
5.392	0.0485760582830466\\
5.394	0.0501100390709297\\
5.406	0.0501100390709297\\
5.408	0.0516440198588164\\
5.42	0.0516440198588164\\
5.422	0.0531780006467031\\
5.436	0.0531780006467031\\
5.438	0.0547119814345862\\
5.45	0.0547119814345862\\
5.452	0.0562459622224729\\
5.464	0.0562459622224729\\
5.466	0.0577799430103596\\
5.478	0.0577799430103596\\
5.48	0.0593139237982463\\
5.494	0.0593139237982463\\
5.496	0.0608479045861294\\
5.508	0.0608479045861294\\
5.51	0.0623818853740161\\
5.524	0.0623818853740161\\
5.526	0.0639158661619028\\
5.538	0.0639158661619028\\
5.54	0.065449846949786\\
5.552	0.065449846949786\\
5.554	0.0669838277376726\\
5.566	0.0669838277376726\\
5.568	0.0685178085255593\\
5.58	0.0685178085255593\\
5.582	0.070051789313446\\
5.594	0.070051789313446\\
5.596	0.0715857701013292\\
5.606	0.0715857701013292\\
5.608	0.0731197508892159\\
5.62	0.0731197508892159\\
5.622	0.0746537316771025\\
5.634	0.0746537316771025\\
5.636	0.0761877124649857\\
5.648	0.0761877124649857\\
5.65	0.0777216932528724\\
5.662	0.0777216932528724\\
5.664	0.0792556740407591\\
5.674	0.0792556740407591\\
5.676	0.0807896548286422\\
5.688	0.0807896548286422\\
5.69	0.0823236356165289\\
5.7	0.0823236356165289\\
5.702	0.0838576164044156\\
5.712	0.0838576164044156\\
5.714	0.0853915971923023\\
5.724	0.0853915971923023\\
5.726	0.0869255779801854\\
5.738	0.0869255779801854\\
5.74	0.0884595587680721\\
5.748	0.0884595587680721\\
5.75	0.0899935395559588\\
5.76	0.0899935395559588\\
5.762	0.0915275203438419\\
5.774	0.0915275203438419\\
5.776	0.0930615011317286\\
5.786	0.0930615011317286\\
5.788	0.0945954819196153\\
5.798	0.0945954819196153\\
5.8	0.0961294627074984\\
5.81	0.0961294627074984\\
5.812	0.0976634434953851\\
5.822	0.0976634434953851\\
5.824	0.0991974242832718\\
5.834	0.0991974242832718\\
5.836	0.100731405071159\\
5.846	0.100731405071159\\
5.848	0.102265385859042\\
5.858	0.102265385859042\\
5.86	0.103799366646928\\
5.868	0.103799366646928\\
5.87	0.105333347434815\\
5.88	0.105333347434815\\
5.882	0.106867328222698\\
5.892	0.106867328222698\\
5.894	0.108401309010585\\
5.904	0.108401309010585\\
5.906	0.109935289798472\\
5.916	0.109935289798472\\
5.918	0.111469270586358\\
5.928	0.111469270586358\\
5.93	0.113003251374241\\
5.94	0.113003251374241\\
5.942	0.114537232162128\\
5.952	0.114537232162128\\
5.954	0.116071212950015\\
5.964	0.116071212950015\\
5.966	0.117605193737898\\
5.976	0.117605193737898\\
5.978	0.119139174525785\\
5.986	0.119139174525785\\
5.988	0.120673155313671\\
5.996	0.120673155313671\\
5.998	0.122207136101554\\
6.008	0.122207136101554\\
6.01	0.123741116889441\\
6.02	0.123741116889441\\
6.022	0.125275097677328\\
6.032	0.125275097677328\\
6.034	0.126809078465214\\
6.044	0.126809078465214\\
6.046	0.128343059253098\\
6.056	0.128343059253098\\
6.058	0.129877040040984\\
6.068	0.129877040040984\\
6.07	0.131411020828871\\
6.078	0.131411020828871\\
6.08	0.132945001616754\\
6.088	0.132945001616754\\
6.09	0.134478982404641\\
6.1	0.134478982404641\\
6.102	0.136012963192528\\
6.112	0.136012963192528\\
6.114	0.137546943980414\\
6.124	0.137546943980414\\
6.126	0.139080924768297\\
6.136	0.139080924768297\\
6.138	0.140614905556184\\
6.148	0.140614905556184\\
6.15	0.142148886344071\\
6.158	0.142148886344071\\
6.16	0.143682867131954\\
6.17	0.143682867131954\\
6.172	0.145216847919841\\
6.18	0.145216847919841\\
6.182	0.146750828707727\\
6.192	0.146750828707727\\
6.194	0.14828480949561\\
6.206	0.14828480949561\\
6.208	0.149818790283497\\
6.216	0.149818790283497\\
6.218	0.151352771071384\\
6.228	0.151352771071384\\
6.23	0.15288675185927\\
6.24	0.15288675185927\\
6.242	0.154420732647154\\
6.25	0.154420732647154\\
6.252	0.15595471343504\\
6.26	0.15595471343504\\
6.262	0.157488694222927\\
6.272	0.157488694222927\\
6.274	0.15902267501081\\
6.284	0.15902267501081\\
6.286	0.160556655798697\\
6.296	0.160556655798697\\
6.298	0.162090636586584\\
6.306	0.162090636586584\\
6.308	0.163624617374467\\
6.316	0.163624617374467\\
6.318	0.165158598162353\\
6.326	0.165158598162353\\
6.328	0.16669257895024\\
6.338	0.16669257895024\\
6.34	0.168226559738127\\
6.35	0.168226559738127\\
6.352	0.16976054052601\\
6.36	0.16976054052601\\
6.362	0.171294521313897\\
6.372	0.171294521313897\\
6.374	0.172828502101783\\
6.384	0.172828502101783\\
6.386	0.174362482889666\\
6.394	0.174362482889666\\
6.396	0.175896463677553\\
6.404	0.175896463677553\\
6.406	0.17743044446544\\
6.414	0.17743044446544\\
6.416	0.178964425253326\\
6.426	0.178964425253326\\
6.428	0.18049840604121\\
6.436	0.18049840604121\\
6.438	0.182032386829096\\
6.448	0.182032386829096\\
6.45	0.183566367616983\\
6.458	0.183566367616983\\
6.46	0.185100348404866\\
6.466	0.185100348404866\\
6.468	0.186634329192753\\
6.478	0.186634329192753\\
6.48	0.188168309980639\\
6.488	0.188168309980639\\
6.49	0.189702290768523\\
6.5	0.189702290768523\\
6.502	0.191236271556409\\
6.51	0.191236271556409\\
6.512	0.192770252344296\\
6.518	0.192770252344296\\
6.52	0.194304233132183\\
6.528	0.194304233132183\\
6.53	0.195838213920066\\
6.54	0.195838213920066\\
6.542	0.197372194707953\\
6.552	0.197372194707953\\
6.554	0.198906175495839\\
6.562	0.198906175495839\\
6.564	0.200440156283722\\
6.572	0.200440156283722\\
6.574	0.201974137071609\\
6.582	0.201974137071609\\
6.584	0.203508117859496\\
6.592	0.203508117859496\\
6.594	0.205042098647382\\
6.602	0.205042098647382\\
6.604	0.206576079435266\\
6.612	0.206576079435266\\
6.614	0.208110060223152\\
6.622	0.208110060223152\\
6.624	0.209644041011039\\
6.63	0.209644041011039\\
6.632	0.211178021798922\\
6.642	0.211178021798922\\
6.644	0.212712002586809\\
6.652	0.212712002586809\\
6.654	0.214245983374695\\
6.662	0.214245983374695\\
6.664	0.215779964162579\\
6.672	0.215779964162579\\
6.674	0.217313944950465\\
6.682	0.217313944950465\\
6.684	0.218847925738352\\
6.692	0.218847925738352\\
6.694	0.220381906526239\\
6.702	0.220381906526239\\
6.704	0.221915887314122\\
6.714	0.221915887314122\\
6.716	0.223449868102009\\
6.722	0.223449868102009\\
6.724	0.224983848889895\\
6.732	0.224983848889895\\
6.734	0.226517829677778\\
6.744	0.226517829677778\\
6.746	0.228051810465665\\
6.754	0.228051810465665\\
6.756	0.229585791253552\\
6.764	0.229585791253552\\
6.766	0.231119772041435\\
6.774	0.231119772041435\\
6.776	0.232653752829322\\
6.784	0.232653752829322\\
6.786	0.234187733617208\\
6.794	0.234187733617208\\
6.796	0.235721714405095\\
6.806	0.235721714405095\\
6.808	0.237255695192978\\
6.814	0.237255695192978\\
6.816	0.238789675980865\\
6.824	0.238789675980865\\
6.826	0.240323656768751\\
6.834	0.240323656768751\\
6.836	0.241857637556635\\
6.846	0.241857637556635\\
6.848	0.243391618344521\\
6.856	0.243391618344521\\
6.858	0.244925599132408\\
6.866	0.244925599132408\\
6.868	0.246459579920295\\
6.876	0.246459579920295\\
6.878	0.247993560708178\\
6.886	0.247993560708178\\
6.888	0.249527541496064\\
6.898	0.249527541496064\\
6.9	0.251061522283951\\
6.908	0.251061522283951\\
6.91	0.252595503071834\\
6.92	0.252595503071834\\
6.922	0.254129483859721\\
6.93	0.254129483859721\\
6.932	0.255663464647608\\
6.94	0.255663464647608\\
6.942	0.257197445435491\\
6.952	0.257197445435491\\
6.954	0.258731426223378\\
6.962	0.258731426223378\\
6.964	0.260265407011264\\
6.972	0.260265407011264\\
6.974	0.261799387799151\\
6.982	0.261799387799151\\
6.984	0.263333368587034\\
6.994	0.263333368587034\\
6.996	0.264867349374921\\
7.004	0.264867349374921\\
7.006	0.266401330162807\\
7.014	0.266401330162807\\
7.016	0.267935310950691\\
7.024	0.267935310950691\\
7.026	0.269469291738577\\
7.034	0.269469291738577\\
7.036	0.271003272526464\\
7.046	0.271003272526464\\
7.048	0.272537253314351\\
7.056	0.272537253314351\\
7.058	0.274071234102234\\
7.066	0.274071234102234\\
7.068	0.27560521489012\\
7.078	0.27560521489012\\
7.08	0.277139195678007\\
7.09	0.277139195678007\\
7.092	0.27867317646589\\
7.098	0.27867317646589\\
7.1	0.280207157253777\\
7.11	0.280207157253777\\
7.112	0.281741138041664\\
7.12	0.281741138041664\\
7.122	0.283275118829547\\
7.132	0.283275118829547\\
7.134	0.284809099617434\\
7.142	0.284809099617434\\
7.144	0.28634308040532\\
7.154	0.28634308040532\\
7.156	0.287877061193207\\
7.166	0.287877061193207\\
7.168	0.28941104198109\\
7.176	0.28941104198109\\
7.178	0.290945022768977\\
7.186	0.290945022768977\\
7.188	0.292479003556863\\
7.196	0.292479003556863\\
7.198	0.294012984344747\\
7.206	0.294012984344747\\
7.208	0.295546965132633\\
7.218	0.295546965132633\\
7.22	0.29708094592052\\
7.23	0.29708094592052\\
7.232	0.298614926708407\\
7.242	0.298614926708407\\
7.244	0.30014890749629\\
7.252	0.30014890749629\\
7.254	0.301682888284176\\
7.264	0.301682888284176\\
7.266	0.303216869072063\\
7.276	0.303216869072063\\
7.278	0.304750849859946\\
7.286	0.304750849859946\\
7.288	0.306284830647833\\
7.296	0.306284830647833\\
7.298	0.30781881143572\\
7.308	0.30781881143572\\
7.31	0.309352792223603\\
7.318	0.309352792223603\\
7.32	0.310886773011489\\
7.33	0.310886773011489\\
7.332	0.312420753799376\\
7.342	0.312420753799376\\
7.344	0.313954734587263\\
7.354	0.313954734587263\\
7.356	0.315488715375146\\
7.366	0.315488715375146\\
7.368	0.317022696163033\\
7.378	0.317022696163033\\
7.38	0.318556676950919\\
7.388	0.318556676950919\\
7.39	0.320090657738803\\
7.4	0.320090657738803\\
7.402	0.321624638526689\\
7.408	0.321624638526689\\
7.41	0.323158619314576\\
7.42	0.323158619314576\\
7.422	0.324692600102459\\
7.43	0.324692600102459\\
7.432	0.326226580890346\\
7.442	0.326226580890346\\
7.444	0.327760561678232\\
7.454	0.327760561678232\\
7.456	0.329294542466119\\
7.466	0.329294542466119\\
7.468	0.330828523254002\\
7.478	0.330828523254002\\
7.48	0.332362504041889\\
7.49	0.332362504041889\\
7.492	0.333896484829776\\
7.502	0.333896484829776\\
7.504	0.335430465617659\\
7.514	0.335430465617659\\
7.516	0.336964446405545\\
7.526	0.336964446405545\\
7.528	0.338498427193432\\
7.538	0.338498427193432\\
7.54	0.340032407981319\\
7.55	0.340032407981319\\
7.552	0.341566388769202\\
7.564	0.341566388769202\\
7.566	0.343100369557089\\
7.576	0.343100369557089\\
7.578	0.344634350344975\\
7.588	0.344634350344975\\
7.59	0.346168331132858\\
7.6	0.346168331132858\\
7.602	0.347702311920745\\
7.614	0.347702311920745\\
7.616	0.349236292708632\\
7.626	0.349236292708632\\
7.628	0.350770273496515\\
7.638	0.350770273496515\\
7.64	0.352304254284402\\
7.652	0.352304254284402\\
7.654	0.353838235072288\\
7.664	0.353838235072288\\
7.666	0.355372215860175\\
7.678	0.355372215860175\\
7.68	0.356906196648058\\
7.694	0.356906196648058\\
7.696	0.358440177435945\\
7.708	0.358440177435945\\
7.71	0.359974158223832\\
7.722	0.359974158223832\\
7.724	0.361508139011715\\
7.738	0.361508139011715\\
7.74	0.363042119799601\\
7.752	0.363042119799601\\
7.754	0.364576100587488\\
7.766	0.364576100587488\\
7.768	0.366110081375375\\
7.78	0.366110081375375\\
7.782	0.367644062163258\\
7.794	0.367644062163258\\
7.796	0.369178042951145\\
7.81	0.369178042951145\\
7.812	0.370712023739031\\
7.824	0.370712023739031\\
7.826	0.372246004526914\\
7.838	0.372246004526914\\
7.84	0.373779985314801\\
7.852	0.373779985314801\\
7.854	0.375313966102688\\
7.872	0.375313966102688\\
7.874	0.376847946890571\\
7.886	0.376847946890571\\
7.888	0.378381927678458\\
7.902	0.378381927678458\\
7.904	0.379915908466344\\
7.92	0.379915908466344\\
7.922	0.381449889254231\\
7.934	0.381449889254231\\
7.936	0.382983870042114\\
7.954	0.382983870042114\\
7.956	0.384517850830001\\
7.968	0.384517850830001\\
7.97	0.386051831617888\\
7.99	0.386051831617888\\
7.992	0.387585812405771\\
8.004	0.387585812405771\\
8.006	0.389119793193657\\
8.026	0.389119793193657\\
8.028	0.390653773981544\\
8.042	0.390653773981544\\
8.044	0.392187754769427\\
8.062	0.392187754769427\\
8.064	0.393721735557314\\
8.084	0.393721735557314\\
8.086	0.395255716345201\\
8.106	0.395255716345201\\
8.108	0.396789697133087\\
8.128	0.396789697133087\\
8.13	0.39832367792097\\
8.152	0.39832367792097\\
8.154	0.399857658708857\\
8.174	0.399857658708857\\
8.176	0.401391639496744\\
8.198	0.401391639496744\\
8.2	0.402925620284627\\
8.224	0.402925620284627\\
8.226	0.404459601072514\\
8.256	0.404459601072514\\
8.258	0.4059935818604\\
8.286	0.4059935818604\\
8.288	0.407527562648287\\
8.29	0.4059935818604\\
8.292	0.4059935818604\\
8.294	0.407527562648287\\
8.318	0.407527562648287\\
8.32	0.40906154343617\\
8.358	0.40906154343617\\
8.36	0.410595524224057\\
8.362	0.410595524224057\\
8.364	0.40906154343617\\
8.366	0.40906154343617\\
8.368	0.410595524224057\\
8.42	0.410595524224057\\
8.422	0.412129505011944\\
8.424	0.412129505011944\\
8.426	0.410595524224057\\
8.428	0.410595524224057\\
8.43	0.412129505011944\\
8.518	0.412129505011944\\
8.52	0.413663485799827\\
8.524	0.413663485799827\\
8.526	0.412129505011944\\
8.53	0.412129505011944\\
8.532	0.413663485799827\\
8.536	0.413663485799827\\
8.538	0.412129505011944\\
8.542	0.412129505011944\\
8.544	0.413663485799827\\
8.548	0.413663485799827\\
8.55	0.412129505011944\\
8.556	0.412129505011944\\
8.558	0.413663485799827\\
8.56	0.413663485799827\\
8.562	0.412129505011944\\
8.684	0.412129505011944\\
8.686	0.410595524224057\\
8.688	0.410595524224057\\
8.69	0.412129505011944\\
8.696	0.412129505011944\\
8.698	0.410595524224057\\
8.7	0.410595524224057\\
8.702	0.412129505011944\\
8.706	0.412129505011944\\
8.708	0.410595524224057\\
8.768	0.410595524224057\\
8.77	0.40906154343617\\
8.772	0.40906154343617\\
8.774	0.410595524224057\\
8.778	0.410595524224057\\
8.78	0.40906154343617\\
8.864	0.40906154343617\\
8.866	0.407527562648287\\
8.958	0.407527562648287\\
8.96	0.4059935818604\\
8.962	0.407527562648287\\
8.968	0.407527562648287\\
8.97	0.4059935818604\\
8.974	0.4059935818604\\
8.976	0.407527562648287\\
8.978	0.407527562648287\\
8.98	0.4059935818604\\
9.112	0.4059935818604\\
9.114	0.404459601072514\\
9.116	0.404459601072514\\
9.118	0.4059935818604\\
9.122	0.4059935818604\\
9.124	0.404459601072514\\
9.186	0.404459601072514\\
9.188	0.402925620284627\\
9.19	0.402925620284627\\
9.192	0.404459601072514\\
9.196	0.404459601072514\\
9.198	0.402925620284627\\
9.258	0.402925620284627\\
9.26	0.401391639496744\\
9.264	0.401391639496744\\
9.266	0.402925620284627\\
9.268	0.402925620284627\\
9.27	0.401391639496744\\
9.308	0.401391639496744\\
9.31	0.399857658708857\\
9.348	0.399857658708857\\
9.35	0.39832367792097\\
9.384	0.39832367792097\\
9.386	0.396789697133087\\
9.416	0.396789697133087\\
9.418	0.395255716345201\\
9.42	0.395255716345201\\
9.422	0.396789697133087\\
9.424	0.395255716345201\\
9.45	0.395255716345201\\
9.452	0.393721735557314\\
9.476	0.393721735557314\\
9.478	0.392187754769427\\
9.512	0.392187754769427\\
9.514	0.390653773981544\\
9.536	0.390653773981544\\
9.538	0.389119793193657\\
9.562	0.389119793193657\\
9.564	0.387585812405771\\
9.598	0.387585812405771\\
9.6	0.386051831617888\\
9.634	0.386051831617888\\
9.636	0.384517850830001\\
9.66	0.384517850830001\\
9.662	0.382983870042114\\
9.696	0.382983870042114\\
9.698	0.381449889254231\\
9.722	0.381449889254231\\
9.724	0.379915908466344\\
9.726	0.379915908466344\\
9.728	0.381449889254231\\
9.732	0.381449889254231\\
9.734	0.379915908466344\\
9.77	0.379915908466344\\
9.772	0.378381927678458\\
9.796	0.378381927678458\\
9.798	0.376847946890571\\
9.8	0.376847946890571\\
9.802	0.378381927678458\\
9.806	0.378381927678458\\
9.808	0.376847946890571\\
9.832	0.376847946890571\\
9.834	0.375313966102688\\
9.87	0.375313966102688\\
9.872	0.373779985314801\\
9.906	0.373779985314801\\
9.908	0.372246004526914\\
9.932	0.372246004526914\\
9.934	0.370712023739031\\
9.936	0.370712023739031\\
9.938	0.372246004526914\\
9.942	0.372246004526914\\
9.944	0.370712023739031\\
9.968	0.370712023739031\\
9.97	0.369178042951145\\
9.994	0.369178042951145\\
9.996	0.367644062163258\\
10.02	0.367644062163258\\
10.022	0.366110081375375\\
10.044	0.366110081375375\\
10.046	0.364576100587488\\
10.07	0.364576100587488\\
10.072	0.363042119799601\\
10.094	0.363042119799601\\
10.096	0.361508139011715\\
10.118	0.361508139011715\\
10.12	0.359974158223832\\
10.136	0.359974158223832\\
10.138	0.358440177435945\\
10.16	0.358440177435945\\
10.162	0.356906196648058\\
10.184	0.356906196648058\\
10.186	0.355372215860175\\
10.206	0.355372215860175\\
10.208	0.353838235072288\\
10.224	0.353838235072288\\
10.226	0.352304254284402\\
10.248	0.352304254284402\\
10.25	0.350770273496515\\
10.27	0.350770273496515\\
10.272	0.349236292708632\\
10.288	0.349236292708632\\
10.29	0.347702311920745\\
10.31	0.347702311920745\\
10.312	0.346168331132858\\
10.332	0.346168331132858\\
10.334	0.344634350344975\\
10.358	0.344634350344975\\
10.36	0.343100369557089\\
10.374	0.343100369557089\\
10.376	0.341566388769202\\
10.398	0.341566388769202\\
10.4	0.340032407981319\\
10.42	0.340032407981319\\
10.422	0.338498427193432\\
10.444	0.338498427193432\\
10.446	0.336964446405545\\
10.46	0.336964446405545\\
10.462	0.335430465617659\\
10.484	0.335430465617659\\
10.486	0.333896484829776\\
10.51	0.333896484829776\\
10.512	0.332362504041889\\
10.544	0.332362504041889\\
10.546	0.330828523254002\\
10.568	0.330828523254002\\
10.57	0.329294542466119\\
10.592	0.329294542466119\\
10.594	0.327760561678232\\
10.616	0.327760561678232\\
10.618	0.326226580890346\\
10.642	0.326226580890346\\
10.644	0.324692600102459\\
10.668	0.324692600102459\\
10.67	0.323158619314576\\
10.694	0.323158619314576\\
10.696	0.321624638526689\\
10.718	0.321624638526689\\
10.72	0.320090657738803\\
10.742	0.320090657738803\\
10.744	0.318556676950919\\
10.764	0.318556676950919\\
10.766	0.317022696163033\\
10.786	0.317022696163033\\
10.788	0.315488715375146\\
10.808	0.315488715375146\\
10.81	0.313954734587263\\
10.83	0.313954734587263\\
10.832	0.312420753799376\\
10.852	0.312420753799376\\
10.854	0.310886773011489\\
10.872	0.310886773011489\\
10.874	0.309352792223603\\
10.89	0.309352792223603\\
10.892	0.30781881143572\\
10.91	0.30781881143572\\
10.912	0.306284830647833\\
10.928	0.306284830647833\\
10.93	0.304750849859946\\
10.948	0.304750849859946\\
10.95	0.303216869072063\\
10.968	0.303216869072063\\
10.97	0.301682888284176\\
10.988	0.301682888284176\\
10.99	0.30014890749629\\
11.008	0.30014890749629\\
11.01	0.298614926708407\\
11.03	0.298614926708407\\
11.032	0.29708094592052\\
11.046	0.29708094592052\\
11.048	0.295546965132633\\
11.068	0.295546965132633\\
11.07	0.294012984344747\\
11.086	0.294012984344747\\
11.088	0.292479003556863\\
11.108	0.292479003556863\\
11.11	0.290945022768977\\
11.13	0.290945022768977\\
11.132	0.28941104198109\\
11.152	0.28941104198109\\
11.154	0.287877061193207\\
11.172	0.287877061193207\\
11.174	0.28634308040532\\
11.196	0.28634308040532\\
11.198	0.284809099617434\\
11.216	0.284809099617434\\
11.218	0.283275118829547\\
11.238	0.283275118829547\\
11.24	0.281741138041664\\
11.262	0.281741138041664\\
11.264	0.280207157253777\\
11.286	0.280207157253777\\
11.288	0.27867317646589\\
11.31	0.27867317646589\\
11.312	0.277139195678007\\
11.336	0.277139195678007\\
11.338	0.27560521489012\\
11.358	0.27560521489012\\
11.36	0.274071234102234\\
11.38	0.274071234102234\\
11.382	0.272537253314351\\
11.402	0.272537253314351\\
11.404	0.271003272526464\\
11.428	0.271003272526464\\
11.43	0.269469291738577\\
11.452	0.269469291738577\\
11.454	0.267935310950691\\
11.472	0.267935310950691\\
11.474	0.266401330162807\\
11.494	0.266401330162807\\
11.496	0.264867349374921\\
11.518	0.264867349374921\\
11.52	0.263333368587034\\
11.536	0.263333368587034\\
11.538	0.261799387799151\\
11.56	0.261799387799151\\
11.562	0.260265407011264\\
11.58	0.260265407011264\\
11.582	0.258731426223378\\
11.6	0.258731426223378\\
11.602	0.257197445435491\\
11.624	0.257197445435491\\
11.626	0.255663464647608\\
11.642	0.255663464647608\\
11.644	0.254129483859721\\
11.662	0.254129483859721\\
11.664	0.252595503071834\\
11.68	0.252595503071834\\
11.682	0.251061522283951\\
11.704	0.251061522283951\\
11.706	0.249527541496064\\
11.724	0.249527541496064\\
11.726	0.247993560708178\\
11.742	0.247993560708178\\
11.744	0.246459579920295\\
11.764	0.246459579920295\\
11.766	0.244925599132408\\
11.782	0.244925599132408\\
11.784	0.243391618344521\\
11.806	0.243391618344521\\
11.808	0.241857637556635\\
11.826	0.241857637556635\\
11.828	0.240323656768751\\
11.848	0.240323656768751\\
11.85	0.238789675980865\\
11.87	0.238789675980865\\
11.872	0.237255695192978\\
11.894	0.237255695192978\\
11.896	0.235721714405095\\
11.912	0.235721714405095\\
11.914	0.234187733617208\\
11.936	0.234187733617208\\
11.938	0.232653752829322\\
11.96	0.232653752829322\\
11.962	0.231119772041435\\
11.986	0.231119772041435\\
11.988	0.229585791253552\\
12.01	0.229585791253552\\
12.012	0.228051810465665\\
12.034	0.228051810465665\\
12.036	0.226517829677778\\
12.058	0.226517829677778\\
12.06	0.224983848889895\\
12.084	0.224983848889895\\
12.086	0.223449868102009\\
12.11	0.223449868102009\\
12.112	0.221915887314122\\
12.136	0.221915887314122\\
12.138	0.220381906526239\\
12.16	0.220381906526239\\
12.162	0.218847925738352\\
12.184	0.218847925738352\\
12.186	0.217313944950465\\
12.208	0.217313944950465\\
12.21	0.215779964162579\\
12.234	0.215779964162579\\
12.236	0.214245983374695\\
12.258	0.214245983374695\\
12.26	0.212712002586809\\
12.282	0.212712002586809\\
12.284	0.211178021798922\\
12.306	0.211178021798922\\
12.308	0.209644041011039\\
12.33	0.209644041011039\\
12.332	0.208110060223152\\
12.352	0.208110060223152\\
12.354	0.206576079435266\\
12.376	0.206576079435266\\
12.378	0.205042098647382\\
12.398	0.205042098647382\\
12.4	0.203508117859496\\
12.424	0.203508117859496\\
12.426	0.201974137071609\\
12.448	0.201974137071609\\
12.45	0.200440156283722\\
12.468	0.200440156283722\\
12.47	0.198906175495839\\
12.49	0.198906175495839\\
12.492	0.197372194707953\\
12.516	0.197372194707953\\
12.518	0.195838213920066\\
12.54	0.195838213920066\\
12.542	0.194304233132183\\
12.562	0.194304233132183\\
12.564	0.192770252344296\\
12.588	0.192770252344296\\
12.59	0.191236271556409\\
12.612	0.191236271556409\\
12.614	0.189702290768523\\
12.636	0.189702290768523\\
12.638	0.188168309980639\\
12.662	0.188168309980639\\
12.664	0.186634329192753\\
12.688	0.186634329192753\\
12.69	0.185100348404866\\
12.718	0.185100348404866\\
12.72	0.183566367616983\\
12.748	0.183566367616983\\
12.75	0.182032386829096\\
12.776	0.182032386829096\\
12.778	0.18049840604121\\
12.8	0.18049840604121\\
12.802	0.178964425253326\\
12.83	0.178964425253326\\
12.832	0.17743044446544\\
12.862	0.17743044446544\\
12.864	0.175896463677553\\
12.896	0.175896463677553\\
12.898	0.174362482889666\\
12.924	0.174362482889666\\
12.926	0.172828502101783\\
12.952	0.172828502101783\\
12.954	0.171294521313897\\
12.98	0.171294521313897\\
12.982	0.16976054052601\\
13.018	0.16976054052601\\
13.02	0.168226559738127\\
13.046	0.168226559738127\\
13.048	0.16669257895024\\
13.074	0.16669257895024\\
13.076	0.165158598162353\\
13.102	0.165158598162353\\
13.104	0.163624617374467\\
13.134	0.163624617374467\\
13.136	0.162090636586584\\
13.164	0.162090636586584\\
13.166	0.160556655798697\\
13.194	0.160556655798697\\
13.196	0.15902267501081\\
13.22	0.15902267501081\\
13.222	0.157488694222927\\
13.25	0.157488694222927\\
13.252	0.15595471343504\\
13.28	0.15595471343504\\
13.282	0.154420732647154\\
13.31	0.154420732647154\\
13.312	0.15288675185927\\
13.338	0.15288675185927\\
13.34	0.151352771071384\\
13.366	0.151352771071384\\
13.368	0.149818790283497\\
13.398	0.149818790283497\\
13.4	0.14828480949561\\
13.432	0.14828480949561\\
13.434	0.146750828707727\\
13.46	0.146750828707727\\
13.462	0.145216847919841\\
13.492	0.145216847919841\\
13.494	0.143682867131954\\
13.524	0.143682867131954\\
13.526	0.142148886344071\\
13.558	0.142148886344071\\
13.56	0.140614905556184\\
13.592	0.140614905556184\\
13.594	0.139080924768297\\
13.626	0.139080924768297\\
13.628	0.137546943980414\\
13.658	0.137546943980414\\
13.66	0.136012963192528\\
13.698	0.136012963192528\\
13.7	0.134478982404641\\
13.732	0.134478982404641\\
13.734	0.132945001616754\\
13.766	0.132945001616754\\
13.768	0.131411020828871\\
13.802	0.131411020828871\\
13.804	0.129877040040984\\
13.842	0.129877040040984\\
13.844	0.128343059253098\\
13.876	0.128343059253098\\
13.878	0.126809078465214\\
13.912	0.126809078465214\\
13.914	0.125275097677328\\
13.95	0.125275097677328\\
13.952	0.123741116889441\\
13.988	0.123741116889441\\
13.99	0.122207136101554\\
14.024	0.122207136101554\\
14.026	0.120673155313671\\
14.064	0.120673155313671\\
14.066	0.119139174525785\\
14.098	0.119139174525785\\
14.1	0.117605193737898\\
14.136	0.117605193737898\\
14.138	0.116071212950015\\
14.176	0.116071212950015\\
14.178	0.114537232162128\\
14.216	0.114537232162128\\
14.218	0.113003251374241\\
14.254	0.113003251374241\\
14.256	0.111469270586358\\
14.296	0.111469270586358\\
14.298	0.109935289798472\\
14.338	0.109935289798472\\
14.34	0.108401309010585\\
14.38	0.108401309010585\\
14.382	0.106867328222698\\
14.418	0.106867328222698\\
14.42	0.105333347434815\\
14.462	0.105333347434815\\
14.464	0.103799366646928\\
14.512	0.103799366646928\\
14.514	0.102265385859042\\
14.552	0.102265385859042\\
14.554	0.100731405071159\\
14.59	0.100731405071159\\
14.592	0.0991974242832718\\
14.644	0.0991974242832718\\
14.646	0.0976634434953851\\
14.688	0.0976634434953851\\
14.69	0.0961294627074984\\
14.734	0.0961294627074984\\
14.736	0.0945954819196153\\
14.79	0.0945954819196153\\
14.792	0.0930615011317286\\
14.834	0.0930615011317286\\
14.836	0.0915275203438419\\
14.886	0.0915275203438419\\
14.888	0.0899935395559588\\
14.946	0.0899935395559588\\
14.948	0.0884595587680721\\
14.99	0.0884595587680721\\
14.992	0.0869255779801854\\
15.046	0.0869255779801854\\
15.048	0.0853915971923023\\
15.098	0.0853915971923023\\
15.1	0.0838576164044156\\
15.154	0.0838576164044156\\
15.156	0.0823236356165289\\
15.204	0.0823236356165289\\
15.206	0.0807896548286422\\
15.258	0.0807896548286422\\
15.26	0.0792556740407591\\
15.312	0.0792556740407591\\
15.314	0.0777216932528724\\
15.368	0.0777216932528724\\
15.37	0.0761877124649857\\
15.432	0.0761877124649857\\
15.434	0.0746537316771025\\
15.49	0.0746537316771025\\
15.492	0.0731197508892159\\
15.562	0.0731197508892159\\
15.564	0.0715857701013292\\
15.624	0.0715857701013292\\
15.626	0.070051789313446\\
15.69	0.070051789313446\\
15.692	0.0685178085255593\\
15.756	0.0685178085255593\\
15.758	0.0669838277376726\\
15.816	0.0669838277376726\\
15.818	0.065449846949786\\
15.886	0.065449846949786\\
15.888	0.0639158661619028\\
15.948	0.0639158661619028\\
15.95	0.0623818853740161\\
16.026	0.0623818853740161\\
16.028	0.0608479045861294\\
16.102	0.0608479045861294\\
16.104	0.0593139237982463\\
16.188	0.0593139237982463\\
16.19	0.0577799430103596\\
16.262	0.0577799430103596\\
16.264	0.0562459622224729\\
16.348	0.0562459622224729\\
16.35	0.0547119814345862\\
16.436	0.0547119814345862\\
16.438	0.0531780006467031\\
16.54	0.0531780006467031\\
16.542	0.0516440198588164\\
16.642	0.0516440198588164\\
16.644	0.0501100390709297\\
16.73	0.0501100390709297\\
16.732	0.0485760582830466\\
16.816	0.0485760582830466\\
16.818	0.0470420774951599\\
16.89	0.0470420774951599\\
16.892	0.0455080967072732\\
16.974	0.0455080967072732\\
16.976	0.04397411591939\\
17.084	0.04397411591939\\
17.086	0.0424401351315034\\
17.202	0.0424401351315034\\
17.204	0.0409061543436167\\
17.342	0.0409061543436167\\
17.344	0.03937217355573\\
17.456	0.03937217355573\\
17.458	0.0378381927678468\\
17.57	0.0378381927678468\\
17.572	0.0363042119799601\\
17.698	0.0363042119799601\\
17.7	0.0347702311920735\\
17.814	0.0347702311920735\\
17.816	0.0332362504041903\\
17.938	0.0332362504041903\\
17.94	0.0317022696163036\\
18.068	0.0317022696163036\\
18.07	0.0301682888284169\\
18.28	0.0301682888284169\\
18.282	0.0286343080405302\\
18.452	0.0286343080405302\\
18.454	0.0271003272526471\\
18.584	0.0271003272526471\\
18.586	0.0255663464647604\\
18.74	0.0255663464647604\\
18.742	0.0240323656768737\\
19.158	0.0240323656768737\\
19.16	0.0224983848889906\\
19.3	0.0224983848889906\\
19.302	0.0209644041011039\\
19.416	0.0209644041011039\\
19.418	0.0194304233132172\\
19.788	0.0194304233132172\\
19.79	0.0178964425253341\\
19.994	0.0178964425253341\\
};
\addlegendentry{$\text{e}_{\text{measured}}$}

\end{axis}
\end{tikzpicture}%

    \caption{Caption}
\end{figure}