\section{Conclusion}
Using optimization theory it is possible to calculate optimal trajectories for the helicopter to follow, even with non-linear constraints. By including state feedback in the system it is shown that the performance can get quite satisfactory, however the tuning process of the LQ-controller can be quite laborious. Bryson's rule is a great starting point for this process, and in some instances yields good results unaltered. Travel control still results in a slight constant offset in contrast to the desired trajectory. The elevation control was expected to function better without feedback and high penalties, since the basic control layer already contains a designated elevation controller (in contrast to travel which heavily relies on very accurate pitch control). All in all, satisfactory results are still a possibility, but require certain amounts of tuning and tradeoffs. 
